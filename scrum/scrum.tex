\documentclass{article}

%%%
\usepackage{amsmath}
\usepackage{array}
\usepackage[titletoc]{appendix}
\usepackage{booktabs}
\usepackage{caption}
\usepackage{fancyhdr}
\usepackage{float}
\usepackage[hmargin=1in, vmargin=1in]{geometry}
\usepackage{hyperref}
\usepackage{listings}
\usepackage{txfonts}
\usepackage{titlesec}
\usepackage{titling}
\usepackage[svgnames]{xcolor}

%%% Commands
\newcommand{\mytitle}{Scrum Notes}

%%% fancyhdr
\lhead{}
\chead{\mytitle}
\rhead{}
\pagestyle{fancy}

%%% hyperref
\hypersetup{colorlinks,
            citecolor=black,
            filecolor=black,
            linkcolor=black,
            urlcolor=black}

%%% listings
\lstset{% code block formatting
  backgroundcolor=\color{white},
  basicstyle=\normalsize,
  breakatwhitespace=false,
  breaklines=true,
  captionpos=b,
  commentstyle=\color{green},
  escapeinside={\%*}{*},
  frame=single,
  keepspaces=true,
  keywordstyle=\color{blue}\bfseries,
  language=Python,
  otherkeywords={%
    description,
    device-width,
    http-equiv,
    initial-scale
  },
  numbers=left,
  numbersep=5pt,
  numberstyle=\footnotesize\color{gray},
  rulecolor=\color{black},
  showspaces=false,
  stepnumber=1,
  tabsize=2,
}

%%% tabular
\newcolumntype{L}{>{\arraybackslash}m{4in}}

\pagenumbering{roman}

%%%%%%%%%%%%%%%%%%%%%%%%%%%%%%%%%%%%%%%%%%%%%%%%%%%%%%%%%%%%%%%%%%%%%%%%%%%%%%%
\begin{document}

\author{Timothy J. Helton\\720.641.8370\\timothy.j.helton@gmail.com}
\date{\today}
\title{\mytitle}

\maketitle
\newpage

%%%%%%%%%%%%%%%%%%%%
\tableofcontents
\newpage

\listoffigures
\listoftables
\newpage

%%%%%%%%%%%%%%%%%%%%
\pagenumbering{arabic}
\section{Conventional Development}

%%%%%%%%%%
\subsection{Waterfall Model}
Most common process for software development. All requirements must be
identified and defined in the begining, and modifications to these parameters
is costly.

\begin{enumerate}
  \item Requirements
  \item Analysis
  \item Design
  \item Coding
  \item Software Product
\end{enumerate}

%%%%%%%%%%
\subsection{Iterative Incremental Model}
Repeat the Waterfall Model multiple times incorporating new features with each
iteration. The user is not involved in the design process.

%%%%%%%%%%%%%%%%%%%%
\section{Agile Development}
This method is based off the Iterative Incremental Model with a time-boxed
iterative approach.

Goals
\begin{itemize}
  \item faster time to deliver
  \item reduce uncertainty and risk
  \item increase return on investment by focusing on customer value
\end{itemize}

%%%%%%%%%%
\subsection{Agile Manifesto}
"We are uncovering better ways of developing software by doing it and helping
others do it. Through this work, we have come to value:

\begin{itemize}
  \item Individuals and interactions over processes and tools
  \item Working software over comprehensive documentation
  \item Customer collaboration over contract negotiation
  \item Responding to change over following a plan
\end{itemize}

That is, while there is value in the items on the right, we value the items on
the left more."

Authors: Beck, Kent, et al. (2001)

\begin{table}[H]
  \centering
  \begin{tabular}{l L}
    \hline
    \hline
    Individuals and Interactions &
    self-organization and self-motivation of the team members\\
    {} & continuous interaction for work, clarifications, information among the
         team members\\
    Working Software &
    Delivery of working software at short duration intervals helps gain customer
    trust and assurance in the team\\
    Customer collaboration &
    Constant involvement of customer with the development team ensures
    communication of necessary modifications\\
    Responding to change &
    Focus on quick response to the proposed changes, which is made possible with
    short duration iterations\\
    \hline
  \end{tabular}
  \caption{Agile Manifesto Items}
  \label{tab:parts}
\end{table}

%%%%%%%%%%
\subsection{Agile Methodologies}

\begin{itemize}
  \item Dynamic System Development Methodology (DSDM)
  \item Scrum
    \begin{itemize}
      \item Focuses on management of tasks within a team environment
      \item uses iterative incremental method
      \item quick and frequent deliveries
    \end{itemize}
  \item Extreme Programming (XP)
    \begin{itemize}
      \item frequent releases
      \item short development cycles
      \item allows new customer requirements to be adopted
    \end{itemize}
  \item Test-driven Development (TDD)
    \begin{itemize}
      \item test are written first
      \item minimal amount of code to pass the test is written
      \item once software is working code is cleaned up to acceptable standards
    \end{itemize}
  \item Lean
    \begin{itemize}
      \item the expenditure of resources not adding value to the end customer
            are targeted for elimination.
      \item focus on preserving value with less work
    \end{itemize}
  \item Kanban
    \begin{itemize}
      \item system to improve and keep up a high level of production
    \end{itemize}
\end{itemize}

%%%%%%%%%%%%%%%%%%%%
\section{Scrum}
Scrum is a framework that defines certain rules, events and roles to bring in
regularity. Every event in the framework has a maximum time duration.

\subsection{Sprint}
\begin{itemize}
  \item 2 week or 1 month cycles
  \item work to be performed in the Sprint is planned collaboratively by the
        team
  \item daily 15 minute meeting to plan for that day
  \item Sprint review is held at the end of the Sprint
\end{itemize}

\subsection{Roles}
\begin{itemize}
  \item ScrumMaster
    \begin{itemize}
      \item makes the process run smoothly
      \item removes obstacles that impact productivity
      \item organizes the facilitates the critical meetings
    \end{itemize}
  \item Product Owner
    \begin{itemize}
      \item this is a single person not a committee
      \item manages the product backlog
      \item prioritizes the product backlog
      \item optimize the team
    \end{itemize}
  \item Team
    \begin{itemize}
      \item keep team between 5 -- 9 people
        \begin{itemize}
          \item fewer than 5 members decreases interaction and results in
                smaller productivity gains
          \item more than 9 members requires too much coordination
        \end{itemize}
      \item ensures smooth flow of information and the quick resolution of
            issues
      \item maximize opportunities for feedback
    \end{itemize}
\end{itemize}

%%%%%%%%%%%%%%%%%%%%%%%%%%%%%%%%%%%%%%%%%%%%%%%%%%%%%%%%%%%%%%%%%%%%%%%%%%%%%%%
\end{document}
