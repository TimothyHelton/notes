%%%
\subsection{Import}
\begin{lstlisting}
import numpy as np
\end{lstlisting}

%%%
\subsection{Concatenate}
Use this command to combine arrays.
%
\subsubsection{Example}
\begin{lstlisting}
arr = np.arange(9).reshpae(3, 3)

# Create new array doubling the rows by adding two instances of arr.
arr_rows = np.concatenate([arr, arr], axis=0)

# Create new array doubling the columns by adding two instances of arr.
arr_cols = np.concatenate([arr, arr], axis=1)
\end{lstlisting}


%%%
\subsection{Record Array}
%
\subsubsection{Create from list of tuples}
\begin{itemize}
  \item First create an ndarray from a list of tuples
    \begin{itemize}
      \item Must be a list of tuples
        ({\color{red}{list of lists will not work}})
      \item The data type is very import in this case. Each item in the dtype
        corresponds to the item in the tuple.
      \item The last field of the dtype for each column is the size of the data.
        Without this argument the conversion to an array will create multiple
        empty columns in the array.
    \end{itemize}
  \item After the ndarray is created make a view as a recarray.
\end{itemize}
\begin{lstlisting}[basicstyle=\bfseries\footnotesize]
arr = np.ndarray(list_of_tuples,
                 dtype=[(col1_name, dtype1, (1,)), (col2_name, dtype2, (1,))])
rec_arr = arr.view(np.recarray)
\end{lstlisting}
