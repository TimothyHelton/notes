%%%
\subsection{Import}
\begin{lstlisting}
import pandas as pd
\end{lstlisting}

%%%
\subsection{Data Frames}
Data Frames are like numpy record arrays.
\begin{itemize}

  \item \textbf{add} use this method to add two data frames together
    \begin{itemize}
      \item use the {\color{red}{fill\_value=0}} argument to replace NaN with 0
    \end{itemize}

  \item \textbf{columns} List the column names
    \begin{itemize}

      \item access the column data by using attribute notation (handle.name) or
        dictionary notation (handle['name'])

      \item \color{red}{If the column name is more than one word you must use
        the dictionary notation}
    \end{itemize}

  \item \textbf{cumsum} Returns a cumulative sum for each column
    \begin{itemize}

      \item to return the cumulative sum for each row add the argument
        \color{red}{axis=1}
    \end{itemize}

  \item \textbf{columns.names} \textit{list} add column header names to
    displayed output

  \item \textbf{corr} Return the correlation of the data columns.

  \item \textbf{describe} Returns summary statistics for the data frame

  \item \textbf{drop} Remove an index (row or column)
    \begin{itemize}

      \item Example: remove row named "b"
\begin{lstlisting}
df = df.drop["b"]
\end{lstlisting}

      \item Example: remove columun named "b"
\begin{lstlisting}
df = df.drop["b", axis=1]
\end{lstlisting}

      \item Operation is not done in place.  To save the modified data frame
        assign it to a new handle.

      \item Add the argument \color{red}{axis=1} to drop a column
    \end{itemize}

  \item \textbf{dropna} Returns a dataframe with all rows containing any null
    values removed.
    \begin{itemize}

      \item By adding the argument {\color{red}{how='all'}} only the rows that
        contain all null removed.

      \item By adding the argument {\color{red}{axis=1}} the columns will be
        removed in lieu of the rows.

      \item By adding the argument {\color{red}{thresh}} equal to an integer n
        rows or columns with less than n items will be removed.
    \end{itemize}

  \item \textbf{fillna} Fills all the null values with the given value.
    \begin{itemize}

      \item If a {\color{red}{dictionary}} is passed to fillna it is possible
        to assign different fill values based on column.

      \item Argument {\color{red}{inplace=True}} will permanently alter the
        original dataframe.
    \end{itemize}

  \item \textbf{head} Returns the first n rows

  \item \textbf{idxmin} Returns the index of the each column maximum value
    \begin{itemize}

      \item to find index of the maximum values for each of the rows enter
        the argument \color{red}{axis=1}
    \end{itemize}

  \item \textbf{idxmin} Returns the index of the each column minimum value
    \begin{itemize}

      \item to find index of the minimum values for each of the rows enter
        the argument \color{red}{axis=1}
    \end{itemize}

  \item \textbf{index.names} \textit{list} add index header names to the
    displayed output

  \item \textbf{ix} Returns the row data (handle.ix[3])
    \begin{itemize}

      \item \color{red}{if additional rows and columns are added using this
        command it will have the same effect as calling reindex.}
    \end{itemize}

  \item \textbf{max} find the maximum value of each column
    \begin{itemize}

      \item to find the maximum value for the rows enter the argument
        \color{red}{axis=1}
    \end{itemize}

  \item \textbf{min} find the minimum value of each column
    \begin{itemize}

      \item to find the minimum value for the rows enter the argument
        \color{red}{axis=1}
    \end{itemize}

  \item \textbf{pct\_change} returns the item by item percent change along the
    columns for each row.
    \begin{itemize}

      \item to view the percent change across the rows enter the argument
        \color{red}{axis=1}
    \end{itemize}

  \item \textbf{plot} uses matplotlib to plot the each column of the data frame.
    \begin{itemize}

      \item {\color{red}{make sure to import matplotlib before calling this
        method}}

      \item to see the graph call {\color{red}{plt.show()}}

      \item if using Jupyter notebooks call {\color{red}{\%matplotlib inline}}
    \end{itemize}

  \item \textbf{reindex} allows rows or columns to be reindexed
    \begin{itemize}

      \item \color{red}{if no arguments are passed the rows will be reindexed,
        and if the columns argument is passed the columns will be reindexed.}
    \end{itemize}

  \item \textbf{sum} Returns the sum of the columns
    \begin{itemize}

      \item to sum the rows enter the argument \color{red}{axis=1}
    \end{itemize}

  \item \textbf{swaplevel} if multiple indices are used this method will swap
    their order
    \begin{itemize}

      \item to swap the columns indices add the argument {\color{red}{axis=1}}
    \end{itemize}

  \item \textbf{tail} Returns the last n rows
\end{itemize}

%%%
\subsection{io}

%
\subsubsection{data}
\begin{itemize}

  \item \textbf{pd.io.data.get\_data\_yahoo()} may be used to get stock prices
    from yahoo finance
\end{itemize}

%%%
\subsection{Multiple Indices}

%%%%
\subsubsection{Series}
\begin{itemize}

  \item Assigning multiple indices to a Series will cause it to act like a
    DataFrame

  \item \textbf{unstack} Returns a DataFrame of the multi-index Series
\end{itemize}
\begin{lstlisting}
group = sorted([1, 2] * 3)
idx = ['a', 'b', 'c'] * 2
ser = pd.Series(np.random.randn, index=[group, idx])
df = ser.unstack()
\end{lstlisting}

%%%
\subsection{Read Files}
\begin{lstlisting}
df = pd.read_csv
\end{lstlisting}

%%%
\subsection{Read HTML}

%
\subsubsection{Dependencies}
\begin{itemize}

  \item Beautiful Soup

  \item html5lib
\end{itemize}
\begin{lstlisting}
pip install beautifulsoup4
pip install html5lib
\end{lstlisting}

%
\subsubsection{Import read\_html}
\begin{lstlisting}
from pandas import read_html
\end{lstlisting}

%
\subsubsection{Create List of Data Frames}
\begin{lstlisting}
url = 'string_address_here'
df_list = pd.io.html.read_html(url)
\end{lstlisting}

%%%
\subsection{Read From Clipboard}
This function will take anything on the system clipboard and attempt to convert
it into a DataFrame.
\begin{lstlisting}
pd.read_clipboard()
\end{lstlisting}

%%%
\subsection{Series}
This a vector.
\begin{itemize}

  \item \textbf{drop} Remove an index row
    \begin{itemize}

      \item Example row drop: ser.drop['b']

      \item Operation is not done in place.  To save the modified series
        assign it to a new handle.
    \end{itemize}

  \item \textbf{dropna} Removes all null values.

  \item \textbf{isnull} Returns boolean array based on null value.

  \item \textbf{rank} Return the value rank with respect to index.

  \item \textbf{reindex} Will assign a new index list to the series.
    \begin{itemize}

      \item \textbf{fill\textunderscore value} argument to insert a specific
        value for missing data.

      \item \textbf{method} allows functions to define the reindex
        \begin{itemize}

          \item \textbf{ffill} Forward Fill will propagate a known value for
            missing data until a new known value is found.  Then the new value
            is propagated for missing data until the next known value is found.

          \item \textbf{bfill} Backward Fill is the same as Forward Fill only
            the propagation goes backward.

          \item \textbf{nearest} Uses the nearest value to propagate unknown
            values.
        \end{itemize}
    \end{itemize}

  \item \textbf{sort\_index} Will sort the series based on index when no
    arguments are provided.

  \item \textbf{sort\_values} Will sort the series based on value when no
    arguments are provided.

  \item \textbf{unique} Returns the unique values of a series.

  \item \textbf{value\_counts} Returns the number of times each values occurs
    in the series.
\end{itemize}

