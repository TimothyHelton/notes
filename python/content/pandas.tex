%%%
\subsection{Import}
\begin{lstlisting}
import pandas as pd
\end{lstlisting}

%%%
\subsection{Binning}
Binning is capable of placing the count of one list into the spans of another
list. This technique is useful for generating histograms.

\begin{lstlisting}
years = [1900, 1991, 1992, 2008, 2012, 2015, 1987, 1969, 2013, 2008, 1999]
bins = list(range(1960, 2021, 10))
decade_bins = pd.cut(years, bins)

# To view the bins
decade_bins.categories

# To view the values
pd.value_counts(decade_bins)

# To state the number of bins
pd.cut(years, 2, precision=1)
\end{lstlisting}

%%%
\subsection{Data Frames}
Data Frames are like numpy record arrays.

%
\subsubsection{add}
Use this method to add two data frames together
  \begin{itemize}
    \item use the {\color{red}{fill\_value=0}} argument to replace NaN with 0
  \end{itemize}

%
\subsubsection{combine\_first}
Replace null values in the first DataFrame with values from the second
DataFrame.

{\color{red}{Note: If the first DataFrame is missing a row that exists in the
  second DataFrame that row will be added as well.}}
\begin{lstlisting}
df_1.combine_first(df_2)
\end{lstlisting}


%
\subsubsection{columns}
List the column names
  \begin{itemize}

    \item access the column data by using attribute notation (handle.name) or
      dictionary notation (handle['name'])

    \item \color{red}{If the column name is more than one word you must use
      the dictionary notation}
  \end{itemize}

%
\subsubsection{concat}
Combine DataFrames
\begin{lstlisting}
pd.concat([df_1, df_2], ignore_index=True)
\end{lstlisting}

%
\subsubsection{cumsum}
Returns a cumulative sum for each column
  \begin{itemize}

    \item to return the cumulative sum for each row add the argument
      \color{red}{axis=1}
  \end{itemize}

%
\subsubsection{columns.names}
\textit{list} Add column header names to displayed output

%
\subsubsection{corr}
Return the correlation of the data columns.

%
\subsubsection{describe}
Returns summary statistics for the data frame

%
\subsubsection{drop}
Remove an index (row or column)
  \begin{itemize}

    \item Example: remove row named "b"
\begin{lstlisting}
df = df.drop["b"]
\end{lstlisting}

    \item Example: remove columun named "b"
\begin{lstlisting}
df = df.drop["b", axis=1]
\end{lstlisting}

    \item Operation is not done in place.  To save the modified data frame
      assign it to a new handle.

    \item Add the argument \color{red}{axis=1} to drop a column
  \end{itemize}

%
\subsubsection{dropna}
Returns a dataframe with all rows containing any null values removed.
  \begin{itemize}

    \item By adding the argument {\color{red}{how='all'}} only the rows that
      contain all null removed.

    \item By adding the argument {\color{red}{axis=1}} the columns will be
      removed in lieu of the rows.

    \item By adding the argument {\color{red}{thresh}} equal to an integer n
      rows or columns with less than n items will be removed.
  \end{itemize}

%
\subsubsection{drop\_duplicates}
Remove all duplicates.

  \begin{itemize}
    \item calling the method without arguments will drop duplicates based on
      all columns.
\begin{lstlisting}
df.duplicated()
\end{lstlisting}
    \item calling the method with the argument of a column names will only drop
      the duplicate rows based on that column keeping the first instance.
\begin{lstlisting}
df.duplicated(['key_1'])
\end{lstlisting}
  \item providing a column name and the argument take\_last equal to True will
    remove the duplicate rows and keep the last instance.
\begin{lstlisting}
df.duplicated(['key_1', take_last=True])
\end{lstlisting}
  \end{itemize}

%
\subsubsection{duplicated}
Returns a boolean Series stating if the rows are duplicates or not keeping the
first instance.

\begin{lstlisting}
df.duplicated()
\end{lstlisting}

%
\subsubsection{fillna}
Fills all the null values with the given value.
  \begin{itemize}

    \item If a {\color{red}{dictionary}} is passed to fillna it is possible
      to assign different fill values based on column.

    \item Argument {\color{red}{inplace=True}} will permanently alter the
      original dataframe.
  \end{itemize}

%
\subsubsection{head}
Returns the first n rows

%
\subsubsection{idxmin}
Returns the index of the each column maximum value
  \begin{itemize}

    \item to find index of the maximum values for each of the rows enter
      the argument \color{red}{axis=1}
  \end{itemize}

%
\subsubsection{idxmin}
Returns the index of the each column minimum value
  \begin{itemize}

    \item to find index of the minimum values for each of the rows enter
      the argument \color{red}{axis=1}
  \end{itemize}

%
\subsubsection{index.names}
\textit{list} Add index header names to the displayed output

%
\subsubsection{ix}
Returns the row data (handle.ix[3])
  \begin{itemize}

    \item \color{red}{if additional rows and columns are added using this
      command it will have the same effect as calling reindex.}
  \end{itemize}

%
\subsubsection{map}
Allows columns to be added to a DataFrame and the values to be matched to
another column instead of the index.

%
\subsubsection{max}
Find the maximum value of each column
  \begin{itemize}

    \item to find the maximum value for the rows enter the argument
      \color{red}{axis=1}
  \end{itemize}

%
\subsubsection{merge}
  \begin{itemize}
    \item combines two data frames common elements
    \item the column used to merge must exist in each DataFrame
    \item to merge on a specific column use the \textbf{on} argument
    \item to specify how the merge is completed use the \textbf{how} argument
      \begin{itemize}
        \item \textbf{left} - use only keys from the left frame
        \item \textbf{right} - use only keys from the right frame
        \item \textbf{inner} - use union of keys from both frames
        \item \textbf{outter} - use intersection of keys from both frames
      \end{itemize}
    \item if multiple columns have the same name in both DataFrames you may
      define the suffix use with the \textbf{suffixes} argument
  \end{itemize}

%
\subsubsection{min}
Find the minimum value of each column
  \begin{itemize}

    \item to find the minimum value for the rows enter the argument
      \color{red}{axis=1}
  \end{itemize}

%
\subsubsection{pct\_change}
Returns the item by item percent change along the columns for each row.
  \begin{itemize}

    \item to view the percent change across the rows enter the argument
      \color{red}{axis=1}
  \end{itemize}

%
\subsubsection{Pivoting}
Create new views of the data.

\begin{lstlisting}
df.pivot(<index>, <column>, <value)
\end{lstlisting}
%
\subsubsection{plot}
Uses matplotlib to plot the each column of the data frame.
  \begin{itemize}

    \item {\color{red}{make sure to import matplotlib before calling this
      method}}

    \item to see the graph call {\color{red}{plt.show()}}

    \item if using Jupyter notebooks call {\color{red}{\%matplotlib inline}}
  \end{itemize}

%
\subsubsection{reindex}
Allows rows or columns to be reindexed
  \begin{itemize}

    \item \color{red}{if no arguments are passed the rows will be reindexed,
      and if the columns argument is passed the columns will be reindexed.}
  \end{itemize}

%
\subsubsection{sign}
Return a 1 or -1 depending on the sign of the DataFrame value.

%
\subsubsection{sliceing}
  \begin{itemize}
    \item slice a DataFrame line an numpy array
    \item for multiple conditions use an \&
\begin{lstlisting}
df[(df.key1 > value) & (df.key2 <= value)]
\end{lstlisting}
  \end{itemize}

%
\subsubsection{stack}
This method will turn a DataFrame into a hierarchical Series.
{\color{red}{Note: the null values will be removed.}}

\begin{lstlisting}
df.stack()
\end{lstlisting}

%
\subsubsection{sum}
Returns the sum of the columns
  \begin{itemize}

    \item to sum the rows enter the argument \color{red}{axis=1}
  \end{itemize}

%
\subsubsection{swaplevel}
If multiple indices are used this method will swap their order
  \begin{itemize}

    \item to swap the columns indices add the argument {\color{red}{axis=1}}
  \end{itemize}

%
\subsubsection{tail}
Returns the last n rows

%
\subsubsection{take}
Reorder an index based on a list if integers.

\begin{lstlisting}
df.take(new_list)
\end{lstlisting}

%
\subsubsection{transpose}
Swap the index with the columns.

%
\subsubsection{unstack}
\begin{itemize}
  \item Turn a hierarchical Series into a DataFrame.
\end{itemize}

%%%
\subsection{Excel with Python}

%
\subsubsection{Required Installs}
\begin{lstlisting}
pip install xlrd
pip install openpyxl
\end{lstlisting}

%
\subsubsection{Open Excel File}
\begin{lstlisting}
xls_file = pd.ExcelFile('file_name.xlsx')
data_frame = xls_file.parse('Sheet1')
\end{lstlisting}

%%
\subsection{Index}
If the Index class is used to define the index or columns then the name may be
defined in one line.
\begin{lstlisting}
df = pd.DataFrame(np.arange(8).reshape(2, 4),
                  index=pd.Index(['LA', 'SF'], name='city'),
                  columns=pd.Index(list('ABCD'), name='letter'))
\end{lstlisting}

%%%
\subsection{io}

%
\subsubsection{data}
  \begin{itemize}

    \item \textbf{pd.io.data.get\_data\_yahoo()} may be used to get stock prices
      from yahoo finance
  \end{itemize}

%
\subsection{Multiple Indices}

%
\subsubsection{Series}
\begin{itemize}
  \item Assigning multiple indices to a Series will cause it to act like a
    DataFrame

  \item \textbf{unstack} Returns a DataFrame of the multi-index Series
\end{itemize}
\begin{lstlisting}
group = sorted([1, 2] * 3)
idx = ['a', 'b', 'c'] * 2
ser = pd.Series(np.random.randn, index=[group, idx])
df = ser.unstack()
\end{lstlisting}

%%%
\subsection{Outliers}
Find values greater than a perscribed number. In the example 3 was chosen.

\begin{lstlisting}
df[(np.abs(df) > 3).any(1)]
\end{lstlisting}

%%%
\subsection{Read Files}
\begin{lstlisting}
df = pd.read_csv
\end{lstlisting}

%%%
\subsection{Read HTML}

%
\subsubsection{Dependencies}
\begin{itemize}

  \item Beautiful Soup

  \item html5lib
\end{itemize}
\begin{lstlisting}
pip install beautifulsoup4
pip install html5lib
\end{lstlisting}

%
\subsubsection{Import read\_html}
\begin{lstlisting}
from pandas import read_html
\end{lstlisting}

%
\subsubsection{Create List of Data Frames}
\begin{lstlisting}
url = 'string_address_here'
df_list = pd.io.html.read_html(url)
\end{lstlisting}

%%%
\subsection{Read From Clipboard}
This function will take anything on the system clipboard and attempt to convert
it into a DataFrame.
\begin{lstlisting}
pd.read_clipboard()
\end{lstlisting}

%%%
\subsection{Series}
This is a vector.

%
\subsubsection{combine\_first}
Replace null values in one Series with the corresponding values from another
Series.
\begin{lstlisting}
ser_1.combine_first(ser_2)
\end{lstlisting}

%
\subsubsection{concat}
Combine two Series
\begin{lstlisting}
# Combine two Series into a single Series
pd.concat([ser_1, ser_2], axis=0)

# Combine two Series into a single Series with hierarchical index
pd.concat([ser_1, ser_2], axis=0, key=['cat_1', 'cat_2'])

# Combine two Series into a DataFrame
pd.concat([ser_1, ser_2], axis=1)
\end{lstlisting}

%
\subsubsection{drop}
Remove an index row
  \begin{itemize}

    \item Example row drop: ser.drop['b']

    \item Operation is not done in place.  To save the modified series
      assign it to a new handle.
  \end{itemize}

%
\subsubsection{dropna}
Removes all null values.

%
\subsubsection{isnull}
Returns boolean array based on null value.

%
\subsubsection{rank}
Return the value rank with respect to index.

%
\subsubsection{reindex}
Will assign a new index list to the series.
  \begin{itemize}

    \item \textbf{fill\textunderscore value} argument to insert a specific
      value for missing data.

    \item \textbf{method} allows functions to define the reindex
      \begin{itemize}

        \item \textbf{ffill} Forward Fill will propagate a known value for
          missing data until a new known value is found.  Then the new value
          is propagated for missing data until the next known value is found.

        \item \textbf{bfill} Backward Fill is the same as Forward Fill only
          the propagation goes backward.

        \item \textbf{nearest} Uses the nearest value to propagate unknown
          values.
      \end{itemize}
  \end{itemize}

%
\subsubsection{replace}
Where loc is useful for replacing values based on index this method works based
on value.

The following code would replace all values of 10 with 100 and 40 with 400.
\begin{lstlisting}
ser.replace([10, 40], [100, 400])
\end{lstlisting}

%
\subsubsection{sort\_index}
Will sort the series based on index when no arguments are provided.

%
\subsubsection{sort\_values}
Will sort the series based on value when no arguments are provided.

%
\subsubsection{unique}
Returns the unique values of a series.

%
\subsubsection{value\_counts}
Returns the number of times each values occurs in the series.
