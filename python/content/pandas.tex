%%%
\subsection{Import}
\begin{lstlisting}
import pandas as pd
\end{lstlisting}

%%%
\subsection{Data Frames}
Data Frames are like numpy record arrays.
\begin{itemize}
  \item \textbf{add} use this method to add two data frames together
    \begin{itemize}
      \item use the \color{red}{fill\_value=0}
      \color{black}{argument to replce NaN with 0}
    \end{itemize}
  \item \textbf{columns} List the column names
    \begin{itemize}
      \item access the column data by using attribute notation (handle.name) or
        dictionary notation (handle['name'])
      \item \color{red}{If the column name is more than one word you must use
        the dictionary notation}
    \end{itemize}
  \item \textbf{del} Remove a column
  \item \textbf{drop} Remove an index (row or column)
    \begin{itemize}
      \item Example row drop: df.drop['b']
      \item Example column drop: df.drop['b', axis=1]
      \item Operation is not done in place.  To save the modified data frame
        assign it to a new handle.
      \item Add the argument \color{red}{axis=1} to drop a column
    \end{itemize}
  \item \textbf{head} Returns the first n rows
  \item \textbf{ix} Returns the row data (handle.ix[3])
    \begin{itemize}
      \item \color{red}{if additional rows and columns are added using this
        command it will have the same effect as calling reindex.}
    \end{itemize}
  \item \textbf{reindex} allows rows or columns to be reindexed
    \begin{itemize}
      \item \color{red}{if no arguments are passed the rows will be reindexed,
        and if the columns argument is passed the columns will be reindexed.}
    \end{itemize}
  \item \textbf{tail} Returns the last n rows
\end{itemize}

%%%
\subsection{Read From Clipboard}
This function will take anything on the system clipboard and attempt to convert
it into a DataFrame.
\begin{lstlisting}
pd.read_clipboard()
\end{lstlisting}

%%%
\subsection{Series}
This a vector.
\begin{itemize}
  \item \textbf{drop} Remove an index row
    \begin{itemize}
      \item Example row drop: ser.drop['b']
      \item Operation is not done in place.  To save the modified series
        assign it to a new handle.
    \end{itemize}
  \item \textbf{rank} Return the value rank with respect to index.
  \item \textbf{reindex} Will assign a new index list to the series.
    \begin{itemize}
      \item \textbf{fill\textunderscore value} argument to insert a specific
        value for missing data.
      \item \textbf{method} allows functions to define the reindex
        \begin{itemize}
          \item \textbf{ffill} Forward Fill will propagate a known value for
            missing data unil a new known value is found.  Then the new value
            is propagated for missing data until the next known value is found.
          \item \textbf{bfill} Backward Fill is the same as Forward Fill only
            the propagation goes backward.
          \item \textbf{nearest} Uses the nearest value to propagate unkown
            values.
        \end{itemize}
    \end{itemize}
  \item \textbf{sort\_index} Will sort the series based on index when no
    arguments are provided.
  \item \textbf{sort\_values} Will sort the series based on value when no
    arguments are provided.
\end{itemize}
