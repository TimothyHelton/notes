%%%
\subsection{Dependencies}
\begin{lstlisting}
pip install numpy
pip install scipy
pip install matplotlib
pip install pandas
pip install statsmodels
pip install patsy
\end{lstlisting}

%%%
\subsection{Import}
\begin{lstlisting}
import matplotlib as mpl
import matplotlib.pyplot as plt
import numpy as np
import pandas as pd
from scipy import stats
import seaborn as sns
\end{lstlisting}

%%%
\subsection{Box Plot}

\begin{lstlisting}
# Box plot where whiskers include all points
sns.boxplot([data1, data2], whis=np.inf)

# Box plot on the horizontal axis
sns.boxplot([data1, data2], vert=False)
\end{lstlisting}

%%%
\subsection{Carpet Plot (Rug Plot)}
Places a stick for every entry in the data set.

\begin{lstlisting}
sns.rugplot(dataset)
plt.ylim(0, 1)

# Add histogram to carpet plot
plt.hist(dataset, alpha=0.3)
\end{lstlisting}

%%%
\subsection{Combining Plot Styles}

\begin{lstlisting}
sns.distplot(dataset, bins=25, rug=True,
             kde_kws={'color': 'indianred', 'label': 'KDE'},
             hist_kws={'color': 'blue', 'label': 'HIST'},
             rug_kws={'color': 'green', 'label': 'Rug'})
\end{lstlisting}

%%%
\subsection{Correlation Plot}
This plot will make a heat map like plot displaying the correlation values
between variables.
\begin{lstlisting}
sns.corrplot(pandas_dataframe)
\end{lstlisting}

%%%
\subsection{Histograms}

\begin{lstlisting}
# Matplotlib
plt.hist(data_set1, normed=True, alpha=0.5, bins=20)
plt.hist(data_set2, normed=True, alpha=0.5, bins=20)

# Seaborn
sns.jointplot(data_set1, data_set2, kind='hex')
\end{lstlisting}

%%%
\subsection{Kernel Density Estimation Plots (KDE)}

\subsubsection{Long Way}

\begin{lstlisting}
sns.rugplot(dataset)

x_min = dataset.min() - 2
x_max = dataset.max() + 2

x_axis = np.linspace(x_min, x_max, 100)

bandwidth = ((4 * dataset.std()**5) / 3 * len(dataset)))**0.2

kernel_list = []

for data_point in dataset:
    # Create a kernel for each point
    kernel = stats.norm(data_point, bandwidth).pdf(x_axis))
    kernel_list.append(kernel)

    # Scale for plotting
    kernel = kernel / kernel.max()
    kernel = kernel * 0.4

    plt.plot(x_axis, kernel, alpha=0.5)

plt.ylim(0, 1)

kde_sum = np.sum(kernel_list, axis=0)

fig = plt.plot(x_zxis, kde_sum)
sns.rugplot(dataset)

plt.yticks([])
plt.suptitle("Sum of the basis functions")
\end{lstlisting}

%
\subsubsection{Short Way}

\begin{lstlisting}
sns.rugplot(dataset)
sns.kdeplot(dataset, bw=bandwidth, label=label, shade=True)
\end{lstlisting}

%
\subsubsection{Cumulative Distribution Function}

\begin{lstlisting}
sns.kdeplot(dataset, cumulative=True)
\end{lstlisting}

%%%
\subsection{Violin Plot}
Combine a KDE Plot with a Box Plot. This type of plot will distinguish between
data that is symmetric about zero or centered on zero.

\begin{lstlisting}
sns.violinplot([data1, data2], inner='stick')
\end{lstlisting}
