\subsection{Install pip}
%
\subsubsection{Without Internet Connection}
\begin{enumerate}
  \item Download \textbf{get-pip.py}
  \item On a computer with an internet connection create the following wheels.
    \begin{itemize}
      \item pip
      \item setuptools
    \end{itemize}
  \item Move the wheels to the computer without an internet connection.
  \item Call the following command
\begin{lstlisting}[language=bash, numbers=none]
  python get-pip.py --no-index --find-links=WheelHouseDirectory
\end{lstlisting}
\end{enumerate}

%%%
\subsection{Find the Site Packages Installation Directory}
\begin{lstlisting}[language=bash, numbers=none]
  python -c "from distutils.sysconfig
             import get_python_lib;
             print get_python_lib()"
\end{lstlisting}

%%%
\subsection{Install Packages}
%
\subsubsection{Install a Single Package}
\begin{lstlisting}[language=bash, numbers=none]
  pip install PackageName
\end{lstlisting}
%
\subsubsection{Install a Packages From requirments.txt File}
\begin{lstlisting}[language=bash, numbers=none]
  pip install -r requirements.txt
\end{lstlisting}
%
\subsubsection{Install Package in Developer Mode}
This option allows a package to be actively developed while being installed in
a Python interpreter.
\begin{lstlisting}[language=bash, numbers=none]
  pip install -e%*\ *.
\end{lstlisting}

%%%
\subsection{List Outdated Modules}
\begin{lstlisting}[language=bash, numbers=none]
  pip list -o
\end{lstlisting}

%%%
\subsection{Change the Version of an Installed Package}
%
\subsubsection{Upgrade to the Latest Version}
\begin{lstlisting}[language=bash, numbers=none]
  pip install --upgrade PackageName
\end{lstlisting}

\subsubsection{Install a Previous Version}
\begin{lstlisting}[language=bash, numbers=none]
  pip install --upgrade PackageName==Version
\end{lstlisting}

%%%
\subsection{Create a Wheels}
\begin{lstlisting}[language=bash, numbers=none]
  pip wheel --wheel-dir=WheelHouseDirectory
\end{lstlisting}

%%%
\subsection{Package Configuration}
The package configuration may be maintained by a single text file, which is
commonly called requirements.txt. Boolean operations may also be used in these
configuration files. To create a snapshot of the current interpreter use the
following command.
\begin{lstlisting}[language=bash, numbers=none]
  pip freeze > requirements.txt
\end{lstlisting}
