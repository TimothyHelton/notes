\documentclass{article}

%%%
\usepackage{amsmath}
\usepackage[titletoc]{appendix}
\usepackage{booktabs}
\usepackage{caption}
\usepackage{fancyhdr}
\usepackage{float}
\usepackage[hmargin=1in, vmargin=1in]{geometry}
\usepackage{hyperref}
\usepackage{listings}
\usepackage{txfonts}
\usepackage{titlesec}
\usepackage{titling}
\usepackage[svgnames]{xcolor}

%%% Commands
\newcommand{\mytitle}{HTML Notes}

%%% fancyhdr
\lhead{}
\chead{\mytitle}
\rhead{}
\pagestyle{fancy}

%%% hyperref
\hypersetup{colorlinks,
            citecolor=black,
            filecolor=black,
            linkcolor=black,
            urlcolor=black}

%%% listings
\lstset{% code block formatting
  backgroundcolor=\color{white},
  basicstyle=\normalsize,
  breakatwhitespace=false,
  breaklines=true,
  captionpos=b,
  commentstyle=\color{green},
  escapeinside={\%*}{*},
  frame=single,
  keepspaces=true,
  keywordstyle=\color{blue}\bfseries,
  language=HTML,
  otherkeywords={%
    description,
    device-width,
    http-equiv,
    initial-scale
  },
  numbers=left,
  numbersep=5pt,
  numberstyle=\footnotesize\color{gray},
  rulecolor=\color{black},
  showspaces=false,
  stepnumber=1,
  tabsize=2,
}

\pagenumbering{roman}

%%%%%%%%%%%%%%%%%%%%%%%%%%%%%%%%%%%%%%%%%%%%%%%%%%%%%%%%%%%%%%%%%%%%%%%%%%%%%%%
\begin{document}

\author{Timothy J. Helton\\720.641.8370\\timothy.j.helton@gmail.com}
\date{\today}
\title{\mytitle}

\maketitle
\newpage

%%%%%%%%%%%%%%%%%%%%
\tableofcontents
\newpage

\listoffigures
\listoftables
\newpage

%%%%%%%%%%%%%%%%%%%%
\pagenumbering{arabic}
\section{Tags}

%%%%%%%%%%
\subsection{Body Section}
\begin{lstlisting}
<body>
</body>
\end{lstlisting}

%%%%%%%%%%
\subsection{Bold}
\begin{lstlisting}
<p><strong>BOLD Text</strong></p>
\end{lstlisting}

%%%%%%%%%%
\subsection{Bullet List}
\begin{itemize}
  \item ul: unordered list
  \item li: list item
\end{itemize}

\begin{lstlisting}
<ul>
  <li></li>
  <li></li>
    <ul>
      <li></li>
    </ul>
  <li></li>
<ul>
\end{lstlisting}

%%%%%%%%%%
\subsection{Division}
Use this tag to create separate formated sections on a web page.

\begin{lstlisting}
<div>
</div>
\end{lstlisting}

%%%%%%%%%%
\subsection{Document Type}
This the first line of an HTML file.

\begin{lstlisting}
<!doctype html>
\end{lstlisting}

%%%%%%%%%%
\subsection{Drop Down Menu}
\begin{lstlisting}
<select>
  <option>Select From List</option>
  <option>First</option>
  <option>Second</option>
  <option>Third</option>
</select>
\end{lstlisting}

%%%%%%%%%%
\subsection{File Type}
\begin{lstlisting}
<html>
</html>
\end{lstlisting}

%%%%%%%%%%
\subsection{Form}

%%%%%
\subsubsection{input tag}
\begin{itemize}
  \item type=``checkbox''

\begin{lstlisting}
<form>
  <input type=``checkbox'' name=``variable_name'' value=``1''/> Text
</form>
\end{lstlisting}

  \item type=``email''
    \begin{itemize}
      \item This argument will automatically validate if the e-mail address is
            valid or not.
      \item \%40 is the URL encoding for the @ symbol
    \end{itemize}

\begin{lstlisting}
<form>
  <input type=``email'' placeholder=``user@domain'' name=``variable_name''/>
</form>
\end{lstlisting}

  \item type=``radio''
    \begin{itemize}
      \item This argument will add radio button.
      \item The name argument will make the group of radio buttons mutually
            exclusive.
    \end{itemize}

\begin{lstlisting}
<form>
  <input type=``radio'' name=``group_name'' /name=``variable_name''> Title
</form>
\end{lstlisting}

  \item type=``text''
    \begin{itemize}
      \item If you use the argument value=``default text'' the user will have
            to erase the default text before typing. The argument placeholder
            will show up as a hint and will disappear when the user clicks on
            the form.
      \item If you want the user to enter a large section of text use the
            textarea tag in a form.
    \end{itemize}

\begin{lstlisting}
<form>
  <input type=``text'' placeholder=``default text'' /name=``variable_name''/>
</form>
\end{lstlisting}

  \item type=``submit''
    \begin{itemize}
      \item Once button is pressed the data in the forms is converted using
            URL encoding.
      \item The fields on the web page must have the name argument for the
            submit to work.
    \end{itemize}
\begin{lstlisting}
<form>
  <input type=``submit'' value=``Text on Button''/>
</form>
\end{lstlisting}


\end{itemize}

%%%%%
\subsubsection{textarea tag}
This tag allows a larger window for the user to type into.

\begin{lstlisting}
<textarea name=``variable_name''>optional text content</textarea>
\end{lstlisting}

%%%%%%%%%%
\subsection{Head Section}
\begin{lstlisting}
<head>
</head>
\end{lstlisting}

%%%%%%%%%%
\subsection{Headers}
\begin{lstlisting}
<h1></h1>
<h2></h2>
<h3></h3>
<h4></h4>
<h5></h5>
<h6></h6>
\end{lstlisting}

%%%%%%%%%%
\subsection{Horizontal Line}
\begin{lstlisting}
<hr />
\end{lstlisting}

%%%%%%%%%%
\subsection{Links}
\begin{itemize}
  \item destination links use the href argument
\end{itemize}

\begin{lstlisting}
<a href=``full_path_to_external_link''>Name of External Link</a>
\end{lstlisting}

\begin{itemize}
  \item source locations in a web page use the name argument
\end{itemize}

\begin{lstlisting}
<a name=``top'' /a>
<a href=``#Name_of_page_link''>Name of anchor</a>
\end{lstlisting}

%%%%%%%%%%
\subsection{Images}
\begin{itemize}
  \item For a locally stored image:

\begin{lstlisting}
<img src=``images/picture.png''/>
\end{lstlisting}

  \item For an image that is loaded from another location on the internet:

\begin{lstlisting}
<img src=``url ofimage file''/>
\end{lstlisting}

  \item To set the width of an image in pixels:
    \begin{itemize}
      \item Note: Aspect ratio will be maintained.
      \item 100 is about right for a thumbnail
    \end{itemize}

\begin{lstlisting}
<img src=``images/picture.png'' width=``100''/>
\end{lstlisting}

  \item To set the height of an image in pixels:
    \begin{itemize}
      \item Note: Aspect ratio will be maintained.
    \end{itemize}

\begin{lstlisting}
<img src=``images/picture.png'' height=``100''/>
\end{lstlisting}

  \item To set the height and width of an image in pixels:
    \begin{itemize}
      \item Note: Aspect ratio will \color{red}{not} be maintained.
    \end{itemize}

\begin{lstlisting}
<img src=``images/picture.png'' height=``500'' width=``100''/>
\end{lstlisting}

\end{itemize}



%%%%%%%%%%
\subsection{Italics}
\begin{itemize}
  \item em: stands for emphasis
\end{itemize}

\begin{lstlisting}
<p><em>Italics text</em></p>
\end{lstlisting}

%%%%%%%%%%
\subsection{Line Break}
\begin{lstlisting}
<br />
\end{lstlisting}

%%%%%%%%%%
\subsection{Meta Data}
\begin{lstlisting}
<meta charset=``utf-8''/>
<meta http-equiv=``Content-type'' content=``text/html; charset=utf-8''/>
<meta name=``viewport'' content=``width=device-width, initial-scale=1''/>
<meta description= content=``Enter Description Here''/>
\end{lstlisting}

%%%%%%%%%%
\subsection{Numbered List}
\begin{itemize}
  \item ol: ordered list
  \item li: list item
\end{itemize}

\begin{lstlisting}
<ol>
  <li></li>
  <li></li>
    <ol>
      <li></li>
    </ol>
  <li></li>
<ol>
\end{lstlisting}

%%%%%%%%%%
\subsection{Paragraph}
\begin{lstlisting}
<p>
</p>
\end{lstlisting}

%%%%%%%%%%
\subsection{Strike Through}
\begin{lstlisting}
<strike>Strike Through Text</strike>
\end{lstlisting}

%%%%%%%%%%
\subsection{Tables}
\begin{itemize}
  \item can be used for tables
  \item can also be used for layout design
\end{itemize}

\begin{lstlisting}
<table>
  <tr><th>Col0</th><th>Col1</th><th>Col0</th></tr>
  <tr><td>value0</td><td>value1</td><td>value2</td><tr>
</table>
\end{lstlisting}

%%%%%%%%%%
\subsection{Title}
\begin{lstlisting}
<title>This is the Title</title>
\end{lstlisting}

%%%%%%%%%%
\subsection{Underline}
\begin{lstlisting}
<u>Underlined Text</u>
\end{lstlisting}

%%%%%%%%%%%%%%%%%%%%%%%%%%%%%%%%%%%%%%%%%%%%%%%%%%%%%%%%%%%%%%%%%%%%%%%%%%%%%%%
\end{document}
