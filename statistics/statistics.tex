\documentclass{article}

%%%
\usepackage{amsmath}
\usepackage[titletoc]{appendix}
\usepackage{booktabs}
\usepackage{caption}
\usepackage{commath}
\usepackage{fancyhdr}
\usepackage{float}
\usepackage[hmargin=1in, vmargin=1in]{geometry}
\usepackage{hyperref}
\usepackage{listings}
\usepackage{nomencl}
\usepackage{txfonts}
\usepackage{titlesec}
\usepackage{titling}
\usepackage[svgnames]{xcolor}

%%% Commands
\newcommand{\mytitle}{Statistics Notes}

%%% fancyhdr
\lhead{}
\chead{\mytitle}
\rhead{}
\pagestyle{fancy}

%%% hyperref
\hypersetup{colorlinks,
            citecolor=black,
            filecolor=black,
            linkcolor=black,
            urlcolor=black}

%%% listings
\lstset{% code block formatting
  backgroundcolor=\color{white},
  basicstyle=\normalsize,
  breakatwhitespace=false,
  breaklines=true,
  captionpos=b,
  commentstyle=\color{green},
  escapeinside={\%*}{*},
  frame=single,
  keepspaces=true,
  keywordstyle=\color{blue}\bfseries,
  language=HTML,
  otherkeywords={%
    description,
    device-width,
    http-equiv,
    initial-scale
  },
  numbers=left,
  numbersep=5pt,
  numberstyle=\footnotesize\color{gray},
  rulecolor=\color{black},
  showspaces=false,
  stepnumber=1,
  tabsize=2,
}

\pagenumbering{roman}

%%%%%%%%%%%%%%%%%%%%%%%%%%%%%%%%%%%%%%%%%%%%%%%%%%%%%%%%%%%%%%%%%%%%%%%%%%%%%%%
\begin{document}

\author{Timothy J. Helton\\720.641.8370\\timothy.j.helton@gmail.com}
\date{\today}
\title{\mytitle}

\maketitle
\newpage

%%%%%%%%%%%%%%%%%%%%
\tableofcontents
\newpage

\listoffigures
\listoftables

%%%%%%%%%%%%%%%%%%%%
\newpage
\makenomenclature
\renewcommand{\nomname}{Definitions}
\printnomenclature[1.5in]
\hfill

\nomenclature{$\mu$}{average value for population}
\nomenclature{$N$}{population size}
\nomenclature{$n$}{sample size}
\nomenclature{$r$}{Pearson correlation coefficient}
\nomenclature{$s_x$}{Standard Deviation of x}
\nomenclature{$\overline{x}$}{average value for sample}
\nomenclature{covariance}{a descriptive measure of the linear association
                          between two variables}
\nomenclature{negative relationship}{multiple variables have opposite trends}
\nomenclature{population}{all specimens in a set}
\nomenclature{positive relationship}{multiple variables follow the same trends}
\nomenclature{sample}{a subset of the population specimens}
\nomenclature{sampling error}{the difference between corresponding parameters
                              and statistics}
\nomenclature{statistic}{a characteristic of a sample}


%%%%%%%%%%%%%%%%%%%%
\pagenumbering{arabic}
\section{Bivariate Relationships}

%%%
\subsection{Covariance}
\begin{itemize}
  \item A descriptive measure of the linear association between two variables
  \item Only describes {\color{red}{direction}} not magnitude
  \item has no upper or lower boundary
  \item {\color{red}{Covariance is POSITIVE, NEGATIVE or ZERO}}
  \item Positive covariance would be in quadrants I and III
  \item Negative covariance would be in quadrants II and IV
  \item When there is not a relationship between two variables the covariance
    would be equal or near zero
\end{itemize}

\subsubsection{Sample Covariance}
\[ Cov(x,y) = S_{xy} = \frac{\Sigma(x_i - \overline{x})(y_i - \overline{y})}{n-1} \]

\subsubsection{Population Covariance}
\[ \sigma_{xy} = \frac{\Sigma(x_i - \mu_x)(y_i - \mu_y)}{N} \]

%%%
\subsection{Correlation}
\begin{itemize}
  \item Describes the {\color{red}{magnitude and direction}} of an association
  \item Denoted by the lowercase variable r
  \item standardized measure of an association (-1 to 1)
  \item only applicable to {\color{red}{LINEAR}} relationships
  \item \textbf{Correlation is NOT Causation}
  \item correlation strength does not necessarily mean the correlation is
    statistically significant related to sample size
  \item correlation is the covariance between the two variables divided by the
    product of each variables standard deviation
\end{itemize}

\[ r = \frac{Covariance(x,y)}{(Standard Deviation(x))(Standard Deviation(y))} \]
\[ r = \frac{Cov(x,y)}{s_xs_y} \]

\subsubsection{Rule of Thumb for Causation}
If the following relationship is true then causation exists.

\[ \abs{r} \geq \frac{2}{\sqrt{n}} \]

%%%
\subsection{Linear Regression}
\begin{itemize}
  \item If you only have data for one variable then the best predictor for
    future samples would be the mean value( \ $\overline{y}$ / ) of the data.
  \item \textbf{residuals} or \textbf{error} is the measure between individual
    data points to the best fit model.
  \item To compare the fits of models the residuals are squared to emphasize
    the larger contributing samples.
  \item Linear Regression is a comparison of a mathematical model of multiple
    variables to one with fewer independent variables.
\end{itemize}

%%%%%%%%%%%%%%%%%%%%%%%%%%%%%%%%%%%%%%%%%%%%%%%%%%%%%%%%%%%%%%%%%%%%%%%%%%%%%%%
\end{document}
