\documentclass{article}

%%%
\usepackage{amsmath}
\usepackage[titletoc]{appendix}
\usepackage{booktabs}
\usepackage{caption}
\usepackage{fancyhdr}
\usepackage{float}
\usepackage[hmargin=1in, vmargin=1in]{geometry}
\usepackage{hyperref}
\usepackage{listings}
\usepackage{nomencl}
\usepackage{txfonts}
\usepackage{titlesec}
\usepackage{titling}
\usepackage[svgnames]{xcolor}

%%% Commands
\newcommand{\mytitle}{Statistics Notes}

%%% fancyhdr
\lhead{}
\chead{\mytitle}
\rhead{}
\pagestyle{fancy}

%%% hyperref
\hypersetup{colorlinks,
            citecolor=black,
            filecolor=black,
            linkcolor=black,
            urlcolor=black}

%%% listings
\lstset{% code block formatting
  backgroundcolor=\color{white},
  basicstyle=\normalsize,
  breakatwhitespace=false,
  breaklines=true,
  captionpos=b,
  commentstyle=\color{green},
  escapeinside={\%*}{*},
  frame=single,
  keepspaces=true,
  keywordstyle=\color{blue}\bfseries,
  language=HTML,
  otherkeywords={%
    description,
    device-width,
    http-equiv,
    initial-scale
  },
  numbers=left,
  numbersep=5pt,
  numberstyle=\footnotesize\color{gray},
  rulecolor=\color{black},
  showspaces=false,
  stepnumber=1,
  tabsize=2,
}

\pagenumbering{roman}

%%%%%%%%%%%%%%%%%%%%%%%%%%%%%%%%%%%%%%%%%%%%%%%%%%%%%%%%%%%%%%%%%%%%%%%%%%%%%%%
\begin{document}

\author{Timothy J. Helton\\720.641.8370\\timothy.j.helton@gmail.com}
\date{\today}
\title{\mytitle}

\maketitle
\newpage

%%%%%%%%%%%%%%%%%%%%
\tableofcontents
\newpage

\listoffigures
\listoftables

%%%%%%%%%%%%%%%%%%%%
\newpage
\makenomenclature
\renewcommand{\nomname}{Definitions}
\printnomenclature[1.5in]
\hfill

\nomenclature{$\mu$}{average value for population}
\nomenclature{$N$}{population size}
\nomenclature{$n$}{sample size}
\nomenclature{$\overline{x}$}{average value for sample}
\nomenclature{covariance}{a descriptive measure of the linear association
                          between two variables}
\nomenclature{negative relationship}{multiple variables have opposite trends}
\nomenclature{population}{all specimens in a set}
\nomenclature{positive relationship}{multiple variables follow the same trends}
\nomenclature{sample}{a subset of the population specimens}
\nomenclature{sampling error}{the difference between cooresponding parameters
                              and statistics}
\nomenclature{statistic}{a characteristic of a sample}


%%%%%%%%%%%%%%%%%%%%
\pagenumbering{arabic}
\section{Bivariate Relationships}

%%%
\subsection{Covariance}
\begin{itemize}
  \item A descriptive measure of the linear association between two variables
  \item Only describes {\color{red}{direction}} not magnitude
  \item {\color{red}{Covariance is POSITIVE, NEGATIVE or ZERO}}
  \item Positive covariance would be in quadrants I and III
  \item Negative covariance would be in quadrants II and IV
  \item When there is not a relationship between two variables the covariance
    would be equal or near zero
\end{itemize}

\subsubsection{Sample Covarience}
\[ Cov(x,y) = S_{xy} = \frac{\Sigma(x_i - \overline{x})(y_i - \overline{y})}{n-1} \]

\subsubsection{Population Covarience}
\[ \sigma_{xy} = \frac{\Sigma(x_i - \mu_x)(y_i - \mu_y)}{N} \]

%%%
\subsection{Corralation}
\begin{itemize}
  \item Describes the {\color{red}{magnitude}} of an association
\end{itemize}

%%%%%%%%%%%%%%%%%%%%%%%%%%%%%%%%%%%%%%%%%%%%%%%%%%%%%%%%%%%%%%%%%%%%%%%%%%%%%%%
\end{document}
