\documentclass{article}

%%%
\usepackage{amsmath}
\usepackage[titletoc]{appendix}
\usepackage{caption}
\usepackage[superscript, biblabel]{cite}
\usepackage{fancyhdr}
\usepackage{float}
\usepackage[hmargin=1in, vmargin=1in]{geometry}
\usepackage{graphicx}
\usepackage{hyperref}
\usepackage{listings}
\usepackage[lighttt]{lmodern}
\usepackage{nomencl}
\usepackage{subfig}
\usepackage{titlesec}
\usepackage{titling}
\usepackage[svgnames]{xcolor}


%%% Variables
% image height
\newcommand{\imheight}{2.0in}
% input argument
\newcommand{\textarg}[1]{% bold text and change color to blue
  % #1 text
  {\color{blue}{\textbf{#1}}}
}
% scale
\newcommand{\scale}{0.25}
% topic
\newcommand{\topic}{Javascript Programming Language}


%%% fancyhdr
\lhead{\hyperlink{toc}{\topic}}

%%% figure
\restylefloat{figure}

%%% graphicx
\graphicspath{{images/}}

%%% hyperref
\hypersetup{colorlinks,
            citecolor=black,
            filecolor=black,
            linkcolor=black,
            urlcolor=black}

%%% listings
\lstset{%
        backgroundcolor=\color{white},
        basicstyle=\bfseries,
        commentstyle=\color{blue},
        escapeinside={\%*}{*},
        frame=tb,
        language=C,
        keepspaces=true,
        keywordstyle=\color{Peru},
        morestring=[s]{"""}{"""},
        numbers=left,
        numbersep=5pt,
        numberstyle=\tiny,
        showspaces=false,
        showstringspaces=false,
        showtabs=false,
        stringstyle=\color{VioletRed},
        tabsize=4
}

%%% nomencl
\makenomenclature
\renewcommand{\nomname}{Nomenclature}

%%% titlesec
\titleformat{\section}{\color{Blue}\normalfont\Large\bfseries}{\color{Blue}\thesection}{1em}{}
\titleformat{\subsection}{\color{Red}\normalfont\bfseries}{\color{Red}\thesubsection}{1em}{}
\titleclass{\section}{top}
\newcommand\setionbreak{\clearpage}

%%%%%%%%%%%%%%%%%%%%%%%%%%%%%%%%%%%%%%%%%%%%%%%%%%%%%%%%%%%%%%%%%%%%%%%%%%%%%%%
\begin{document}

\pagenumbering{roman}
%%%
\title{\topic \ Notes}
\author{Timothy J. Helton\\720.641.8370\\timothy.j.helton@gmail.com}
\date{\today}

\begin{titlingpage}
  \maketitle
\end{titlingpage}

%%%
\hypertarget{toc}{}
\tableofcontents
\newpage

%%%
\listoffigures
\listoftables
\newpage

%%%
\printnomenclature[0.75in]
\hfill
\nomenclature{numpy}{Base N-dimensional Array Package}
\nomenclature{pip}{Python Package Index}
\nomenclature{scipy}{Python Scientific Computing Package Suite}

\newpage

\pagenumbering{arabic}
\pagestyle{fancy}
%%%%%%%%%%%%%%%%%%%%%%%%%%%%%%%%%%%%%%%%%%%%%%%%%%%%%%%%%%%%
\section{Comment}
Use two forward slashes to designate a comment.

\begin{lstlisting}
// this is comment
\end{lstlisting}

%%%%%%%%%%%%%%%%%%%%%%%%%%%%%%%%%%%%%%%%%%%%%%%%%%%%%%%%%%%%
\section{End of Line}
\begin{itemize}
  \item semicolons are only required when there are multiple statments on the
    same line
\end{itemize}

\begin{lstlisting}
variable = 5; another_variable = 10;
new_variable = 15
\end{lstlisting}

%%%%%%%%%%%%%%%%%%%%%%%%%%%%%%%%%%%%%%%%%%%%%%%%%%%%%%%%%%%%
\section{Images}

%%%
\subsection{Load an image into memory}
\begin{lstlisting}
image = new SimpleImage("name_of_image_file")
\end{lstlisting}

%%%
\subsection{Display Image}
\begin{lstlisting}
print(image)
\end{lstlisting}

%%%
\subsection{Define Handle for an individual pixel}
    \begin{itemize}
      \item integers begin in the upper left hand corner.
      \item integers increase to the right and down
    \end{itemize}
\begin{lstlisting}
pixel = image.getPixel(int, int)
\end{lstlisting}

%%%
\subsection{For Loops}
\begin{itemize}
  \item condition is surrounded in parenthesis
  \item body is contained in french braces
\end{itemize}
\begin{lstlisting}
for (condition) {
    body
}
\end{lstlisting}

%%%
\subsection{Get pixel color}
\begin{itemize}
  \item chose an integer between 0-255
\end{itemize}

\begin{lstlisting}
pixel.getRed(int)
pixel.getGreen(int)
pixel.getGreen(int)
\end{lstlisting}

%%%
\subsection{If Loops}
\begin{itemize}
  \item condition is surrounded in parenthesis
  \item body is contained in french braces
\end{itemize}
\begin{lstlisting}
if (condition) {
    body
}
\end{lstlisting}
%%%
\subsection{Modify pixel color}
\begin{itemize}
  \item chose an integer between 0-255
\end{itemize}
\begin{lstlisting}
pixel.setRed(int)
pixel.setGreen(int)
pixel.setBlue(int)

for (pixel: image) {
    pixel.setBlue(int)
}
\end{lstlisting}

%%%
\subsection{Zoom}
\begin{lstlisting}
image.setZoom(integer)
\end{lstlisting}

%%%%%%%%%%%%%%%%%%%%%%%%%%%%%%%%%%%%%%%%%%%%%%%%%%%%%%%%%%%%
\section{Print}
Use the print command to send output to the screen.

\begin{lstlisting}
print("This will be displayed on the terminal.")
\end{lstlisting}

%%%%%%%%%%%%%%%%%%%%%%%%%%%%%%%%%%%%%%%%%%%%%%%%%%%%%%%%%%%%
\section{Strings}

\begin{itemize}
  \item Quotes: use double quotes to denote a string
\begin{lstlisting}
"This is a string"
\end{lstlisting}

\end{itemize}

%%%%%%%%%%%%%%%%%%%%%%%%%%%%%%%%%%%%%%%%%%%%%%%%%%%%%%%%%%%%
\section{Tables}

%%%
\subsection{Load a table into memory}
\begin{lstlisting}
table = new SimpleTable("table_name.csv")
\end{lstlisting}

%%%
\subsection{Starts With}
Check to see if a row ends with a given string.
\begin{itemize}
  \item the search strings are case sensitive
\end{itemize}
\begin{lstlisting}
table = new SimpleTable("table_name.csv")

for (row: table) {
  if (row.getField("field_name").endsWith("string")) {
    body
  }
}
\end{lstlisting}

%%%
\subsection{Enumerate Table Data}
\begin{lstlisting}
table = new SimpleTable("table_name.csv")

for (row: table) {
  if (row.getField("field_name") == value) {
    body
  }
}
\end{lstlisting}

%%%
\subsection{Starts With}
Check to see if a row starts with a given string.
\begin{itemize}
  \item the search strings are case sensitive
\end{itemize}
\begin{lstlisting}
table = new SimpleTable("table_name.csv")

for (row: table) {
  if (row.getField("field_name").startsWith("string")) {
    body
  }
}
\end{lstlisting}


%%%%%%%%%%%%%%%%%%%%%%%%%%%%%%%%%%%%%%%%%%%%%%%%%%%%%%%%%%%%
\newpage
\bibliographystyle{plain}
%\bibliography{/home/thelton/projects/report_blocks/citations.bib}

%%%%%%%%%%%%%%%%%%%%%%%%%%%%%%%%%%%%%%%%%%%%%%%%%%%%%%%%%%%%%%%%%%%%%%%%%%%%%%%
\end{document}
