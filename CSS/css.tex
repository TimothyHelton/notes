\documentclass{article}

%%%
\usepackage{amsmath}
\usepackage[titletoc]{appendix}
\usepackage{booktabs}
\usepackage{caption}
\usepackage{fancyhdr}
\usepackage{float}
\usepackage[hmargin=1in, vmargin=1in]{geometry}
\usepackage{hyperref}
\usepackage{listings}
\usepackage{txfonts}
\usepackage{titlesec}
\usepackage{titling}
\usepackage[svgnames]{xcolor}

%%% Commands
\newcommand{\mytitle}{Cascading Style Sheets (CSS) Notes}

%%% fancyhdr
\lhead{}
\chead{\mytitle}
\rhead{}
\pagestyle{fancy}

%%% hyperref
\hypersetup{colorlinks,
            citecolor=black,
            filecolor=black,
            linkcolor=black,
            urlcolor=black}

%%% listings
\lstset{% code block formatting
  backgroundcolor=\color{white},
  basicstyle=\normalsize,
  breakatwhitespace=false,
  breaklines=true,
  captionpos=b,
  commentstyle=\color{green},
  escapeinside={\%*}{*},
  frame=single,
  keepspaces=true,
  keywordstyle=\color{blue}\bfseries,
  language=HTML,
  otherkeywords={%
    description,
    device-width,
    http-equiv,
    initial-scale
  },
  numbers=left,
  numbersep=5pt,
  numberstyle=\footnotesize\color{gray},
  rulecolor=\color{black},
  showspaces=false,
  stepnumber=1,
  tabsize=2,
}

\pagenumbering{roman}

%%%%%%%%%%%%%%%%%%%%%%%%%%%%%%%%%%%%%%%%%%%%%%%%%%%%%%%%%%%%%%%%%%%%%%%%%%%%%%%
\begin{document}

\author{Timothy J. Helton\\720.641.8370\\timothy.j.helton@gmail.com}
\date{\today}
\title{\mytitle}

\maketitle
\newpage

%%%%%%%%%%%%%%%%%%%%
\tableofcontents
\newpage

\listoffigures
\listoftables
\newpage

%%%%%%%%%%%%%%%%%%%%
\pagenumbering{arabic}

%%%%%%%%%%%%%%%%%%%%
\section{Comments}

\begin{lstlisting}
/* CSS comments use this syntax */
\end{lstlisting}

%%%%%%%%%%%%%%%%%%%%
\section{class}
\begin{itemize}
  \item start with a period
  \item content is contained in french braces
  \item a class may be applied to a portion of an element using the span tag
  \item it is possible to assign an element more than one tag at a time
\end{itemize}
\begin{lstlisting}
<!doctype html>
<html>
<head>

    .bold {
      font-weight:bold;
    }

    .large {
        color:blue;
        font-size:300%;
    }

    .underline {
        text-decoration:underline;
    }

</head>

<body>
    <p class="large">This is LARGE text.</p>
    <p>This is <span class="underline">underlined</span> text.</p>
    <p>This is <span class="bold underline">underlined</span> text.</p>
</body>
</html>
\end{lstlisting}

%%%%%%%%%%%%%%%%%%%%
\section{div}
\begin{itemize}
  \item div elements are always just boxes that surround the elements within
    them
  \item by creating a CSS div definition it will adjust all div elements
\end{itemize}
\begin{lstlisting}
<!doctype html>
<html>
<head>

    div {
        background-color:blue;
        height:300px;
        width:100px;
        float:right;
    }

</head>

<body>
    <div>
        <p>This is LARGE text.</p>
    </div>
</body>
</html>
\end{lstlisting}

%%%%%%%%%%%%%%%%%%%%
\section{Color Codes}
\begin{itemize}
  \item when using color codes make sure to include the ampersand
\end{itemize}

%%%%%%%%%%%%%%%%%%%%
\section{float}
\begin{itemize}
  \item items floated to the right will stack sequentially to the right
  \item items floated to the left will stack sequentially to the left
  \item to have items floated to the left and right AND then have other items
    below you must add a clearing div
\end{itemize}
\begin{lstlisting}
<!doctype html>
<html>
<head>

    .clear {
        clear:both;
    }

    .floatright {
        background-color:red;
        float:right;
    }

    .floatleft {
        background-color:blue;
        float:left;
    }

</head>

<body>
    <div class="floatright">
        <p>This text is in a red box.</p>
    </div>
    <div class="floatleft">
        <p>This text is in a blue box.</p>
    </div>
    <div class="clear"></div>
    <p>This text will appear below the red box.</p>

</body>
</html>
\end{lstlisting}


%%%%%%%%%%%%%%%%%%%%
\section{id}
\begin{itemize}
  \item start with a ampersand
  \item content is contained in french braces
  \item id= is similar to class= with the exception that the id is only ever
    used once
  \item it is possible to assign an id and a class to the same element
\end{itemize}
\begin{lstlisting}
<!doctype html>
<html>
<head>
    #green {
        color:green;
        font-size:300%;
    }
</head>

<body>
    <p id="green">This is LARGE text.</p>
</body>
</html>
\end{lstlisting}
%%%%%%%%%%%%%%%%%%%%%%%%%%%%%%%%%%%%%%%%%%%%%%%%%%%%%%%%%%%%%%%%%%%%%%%%%%%%%%%
\end{document}
