\documentclass{article}

%%%
\usepackage{amsmath}
\usepackage[titletoc]{appendix}
\usepackage{booktabs}
\usepackage{caption}
\usepackage{fancyhdr}
\usepackage{float}
\usepackage[hmargin=1in, vmargin=1in]{geometry}
\usepackage{hyperref}
\usepackage{listings}
\usepackage{txfonts}
\usepackage{titlesec}
\usepackage{titling}
\usepackage[svgnames]{xcolor}

%%% Commands
\newcommand{\mytitle}{Cascading Style Sheets (CSS) Notes}

%%% fancyhdr
\lhead{}
\chead{\mytitle}
\rhead{}
\pagestyle{fancy}

%%% hyperref
\hypersetup{colorlinks,
            citecolor=black,
            filecolor=black,
            linkcolor=black,
            urlcolor=black}

%%% listings
\lstset{% code block formatting
  backgroundcolor=\color{white},
  basicstyle=\normalsize,
  breakatwhitespace=false,
  breaklines=true,
  captionpos=b,
  commentstyle=\color{green},
  escapeinside={\%*}{*},
  frame=single,
  keepspaces=true,
  keywordstyle=\color{blue}\bfseries,
  language=HTML,
  otherkeywords={%
    description,
    device-width,
    http-equiv,
    initial-scale
  },
  numbers=left,
  numbersep=5pt,
  numberstyle=\footnotesize\color{gray},
  rulecolor=\color{black},
  showspaces=false,
  stepnumber=1,
  tabsize=2,
}

\pagenumbering{roman}

%%%%%%%%%%%%%%%%%%%%%%%%%%%%%%%%%%%%%%%%%%%%%%%%%%%%%%%%%%%%%%%%%%%%%%%%%%%%%%%
\begin{document}

\author{Timothy J. Helton\\720.641.8370\\timothy.j.helton@gmail.com}
\date{\today}
\title{\mytitle}

\maketitle
\newpage

%%%%%%%%%%%%%%%%%%%%
\tableofcontents
\newpage

\listoffigures
\listoftables
\newpage

%%%%%%%%%%%%%%%%%%%%
\pagenumbering{arabic}

%%%%%%%%%%%%%%%%%%%%
\section{Cascade}
CSS files are processed top down. Each selector caries a weight that is used
to determine the style to be applied when there is a conflict. The weight is
written using three hyphenated numbers.\\
The selector furthest to the right is known as the \textit{key selector} and
takes precedence. The remaining selectors are evaluated from right to left.
\\
\\
\begin{tabular}{lc}
  \toprule
  Selector & Weight \\
  \midrule
    type & 0-0-1 \\
    class & 0-1-0 \\
    ID & 1-0-0 \\
  \bottomrule
\end{tabular}
\\
\\
When selector elements are combined the specificity value is just added.\\
\\
The higher the specificity weights rise, the more likely the cascade will break.

%%%%%%%%%%%%%%%%%%%%
\section{Comments}

\begin{lstlisting}
/* CSS comments use this syntax */
\end{lstlisting}

%%%%%%%%%%%%%%%%%%%%
\section{class}
\begin{itemize}
  \item start with a period
  \item content is contained in french braces
  \item a class may be applied to a portion of an element using the span tag
  \item it is possible to assign an element more than one tag at a time
\end{itemize}
\begin{lstlisting}
<!doctype html>
<html>
<head>

    .bold {
      font-weight:bold;
    }

    .large {
        color:blue;
        font-size:300%;
    }

    .underline {
        text-decoration:underline;
    }

</head>

<body>
    <p class="large">This is LARGE text.</p>
    <p>This is <span class="underline">underlined</span> text.</p>
    <p>This is <span class="bold underline">underlined</span> text.</p>
</body>
</html>
\end{lstlisting}

%%%%%%%%%%%%%%%%%%%%
\section{Colors}
CSS uses sRGB (standard red, green, blue) color definitions. To represent the
sRGB colors there are four methods.\\
Hexadecimal or RGB are the most popular and should be preferred. HSL is the
newest and not widely supported.
\begin{itemize}
  \item keywords: These are names like "red" or "green". Most common colors
    have keywrd names.
  \item hexadecimal notation: These names begin with a "\#" and are normally
    followed by six characters. If the six characters are three pairs of
    duplicates the name maybe shortened to just three characters. This notation
    is used because it is possible to define {\color{red}{millions}} of colors.

\begin{lstlisting}
#ff6600
#f60
\end{lstlisting}

  \item RGB: Use the rgb() or rgba() functions with three integers between 0
    and 255 to define the color and one float between 0 and 1 to define
    the transparency.

\begin{lstlisting}
rgb(255, 112, 0)
rgba(255, 112, 0, 0.5)
\end{lstlisting}

  \item HSL: Use the hsl() function to define
    {\color{red}{hue, saturation, and lightness}}.
    \begin{itemize}
      \item hue is an integer between 0 and 360
      \item saturation is a percentage between 0 and 100
      \item lightness is a percentage between 0 and 100
      \item the transparency is a float between 0 and 1
    \end{itemize}

\begin{lstlisting}
hsl(60, 100%, 50%)
hsla(60, 100%, 50%, 0.25)
\end{lstlisting}

\end{itemize}

%%%%%%%%%%%%%%%%%%%%
\section{div}
\begin{itemize}
  \item div elements are always just boxes that surround the elements within
    them
  \item by creating a CSS div definition it will adjust all div elements
\end{itemize}
\begin{lstlisting}
<!doctype html>
<html>
<head>

    div {
        background-color:blue;
        height:300px;
        width:100px;
        float:right;
    }

</head>

<body>
    <div>
        <p>This is LARGE text.</p>
    </div>
</body>
</html>
\end{lstlisting}

%%%%%%%%%%%%%%%%%%%%
\section{float}
\begin{itemize}
  \item items floated to the right will stack sequentially to the right
  \item items floated to the left will stack sequentially to the left
  \item to have items floated to the left and right AND then have other items
    below you must add a clearing div
\end{itemize}
\begin{lstlisting}
<!doctype html>
<html>
<head>

    .clear {
        clear:both;
    }

    .floatright {
        background-color:red;
        float:right;
    }

    .floatleft {
        background-color:blue;
        float:left;
    }

</head>

<body>
    <div class="floatright">
        <p>This text is in a red box.</p>
    </div>
    <div class="floatleft">
        <p>This text is in a blue box.</p>
    </div>
    <div class="clear"></div>
    <p>This text will appear below the red box.</p>

</body>
</html>
\end{lstlisting}


%%%%%%%%%%%%%%%%%%%%
\section{id}
\begin{itemize}
  \item start with a ampersand
  \item content is contained in french braces
  \item id= is similar to class= with the exception that the id is only ever
    used once
  \item it is possible to assign an id and a class to the same element
\end{itemize}
\begin{lstlisting}
<!doctype html>
<html>
<head>
    #green {
        color:green;
        font-size:300%;
    }
</head>

<body>
    <p id="green">This is LARGE text.</p>
</body>
</html>
\end{lstlisting}

%%%%%%%%%%%%%%%%%%%%
\section{Lengths}

%
\subsection{Pixels}
\begin{itemize}
  \item A pixel is equal to $\frac{1}{96}$ of an inch.
  \item use the px suffix to denote pixels
  \item Pixel definitions are stable, but not fancy.
\end{itemize}
\begin{lstlisting}
font-size: 14px;
\end{lstlisting}

%
\subsection{Relative Lengths}
\subsubsection{Percentages}
Percentages set a relative value based on the parent element.

\begin{lstlisting}
.col {
    width: 50%;
}
\end{lstlisting}

%
\subsubsection{Em}
The em unit is a multiplier of the font size of the parent element. When a font
size of the parent element is not defined the next closest parent element is
used.\\
This method is useful for setting padding around text.

\begin{lstlisting}
.banner {
    font-size: 14px;
    width: 5em;
}
\end{lstlisting}

%%%%%%%%%%%%%%%%%%%%%%%%%%%%%%%%%%%%%%%%%%%%%%%%%%%%%%%%%%%%%%%%%%%%%%%%%%%%%%%
\end{document}
